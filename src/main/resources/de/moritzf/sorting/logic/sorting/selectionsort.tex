\documentclass[a4paper]{report}                                     

\usepackage{ngerman}                                                   
\usepackage[utf8]{inputenc}                                                 
\usepackage{amssymb}                                                                                                 
\usepackage{pseudocode} 
\usepackage{listings}
\usepackage{color}
\usepackage{parskip}
\setlength\parindent{0pt}

\definecolor{dkgreen}{rgb}{0,0.6,0}
\definecolor{gray}{rgb}{0.5,0.5,0.5}
\definecolor{mauve}{rgb}{0.58,0,0.82}



\begin{document}

\chapter*{Selectionsort}

\underline{\bf{Laufzeit}}

\begin{tabular}{|l|l|l|}
\hline
Best Case    & Average Case  &  Worst Case\\ \hline
$O(n^2)$ & $O(n^2)$  &  $O(n^2)$  \\ \hline
\end{tabular}

WORST CASE: - 


BEST CASE: -

Da die Laufzeit immer in $O(n^2)$ liegt, ist der Algorithmus zwar ein schönes Lern und Lehrbeispiel für einen Sortieralgorithmus, allerdings sind andere Algorithmen deutlich effizienter.


\underline{\bf{Pseudocode}}



\begin{lstlisting}
Selectionsort
-------------
begin
  for i := 1 to n - 1 do
    begin
      min := i;
      for j := i + 1 to n do 
        if a[j].key < a[min].key
        then min := j;
      t:= a[min]; a[min]:= [i]; a[i] := t;
    end
end
\end{lstlisting}



\end{document} 